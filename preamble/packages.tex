%%%%%%%%%%%%%%%%%%%%%%%%%%%%%%%%%%%%%%%%%%%%%%%%%%%%%%%%%%%%%%%%%%%%%%%%%%%%%%%%%%%%%%%%%%%%%%%%%%%%
%%------------------------------------------ PACKAGES --------------------------------------------%%
%%%%%%%%%%%%%%%%%%%%%%%%%%%%%%%%%%%%%%%%%%%%%%%%%%%%%%%%%%%%%%%%%%%%%%%%%%%%%%%%%%%%%%%%%%%%%%%%%%%%

\usepackage{scrhack} % to avoid warnings

%%%%%%%%%%%%%%%%%%%%%%%%%
%---LANGUAGE PACKAGES---%
%%%%%%%%%%%%%%%%%%%%%%%%%
% Language: ENGLISH (<-change to ngerman for new german.)
\usepackage[english]{babel}
\usepackage[utf8]{inputenc}
\usepackage[T1]{fontenc}

%including pdfs
\usepackage{pdfpages}

%%%%%%%%%%%%%%%%%%%%%%%%%%%%%%%%%%%%%%%%%%%%%%%%%%
% ================== STYLING =================== %
%%%%%%%%%%%%%%%%%%%%%%%%%%%%%%%%%%%%%%%%%%%%%%%%%%

%\newcommand{\nomunit}[1]{%
%        \renewcommand{\nomentryend}{\hspace*{\fill}#1}}


%%%%%%%%%%%%%%%%%%%%%%%%%%
% --- MAKING FIGURES --- %
%%%%%%%%%%%%%%%%%%%%%%%%%%
\usepackage{float} % for positioning of figures (e.g. H )
%\usepackage[dvipsnames]{xcolor} % for more colors
\usepackage{tikz}
\usetikzlibrary{patterns, decorations.pathmorphing}
\usepackage{tikz-3dplot} % used for 3D Plots


%%%%%%%%%%%%%%%%%%%%%%%%%%%%%%%%%%%%%%%
% -------- PHYSICAL NOTATION -------- %
%%%%%%%%%%%%%%%%%%%%%%%%%%%%%%%%%%%%%%%

%%%%%%%%%%%%%%%%%%%%%%%%%%%%%%%%%%%%%%%
% --- MATHEMATICAL REPRESENTATION --- %
%%%%%%%%%%%%%%%%%%%%%%%%%%%%%%%%%%%%%%%
\usepackage{mathtools}
\usepackage{amssymb}
\usepackage{amsmath}
\usepackage{stmaryrd}
\usepackage{slashed}
\usepackage[amsmath,thmmarks]{ntheorem}
\usepackage{framed}     % to frame theorems etc.
\usepackage{mdframed}   % same
\usepackage{dsfont}
\usepackage{xfrac} % use \sfrac for slanted fraction
\usepackage{cancel} % for strike through of canceling terms

%%%%%%%%%%%%%%%%%%%%%%%%%%
% --- NOMENCLAUTURE  --- %
%%%%%%%%%%%%%%%%%%%%%%%%%%
\usepackage[intoc]{nomencl}
\usepackage{siunitx}
\usepackage{ifthen}
% No extra line space between items
%\setlength{\nomitemsep}{-\parsep}

% rename nomenclature
\renewcommand{\nomname}{List of Symbols}

% Divide nomenclature into subgroups
\newlength\preGroupSkip
\setlength{\preGroupSkip}{3.5ex}
\newlength\postGroupSkip
\setlength{\postGroupSkip}{2.3ex}
\newcommand\groupHeading[1]{%
        \par% <- added
        \vspace{\preGroupSkip}%
        \item[\usekomafont{sectioning}\usekomafont{section}#1]%
        \hspace*{\leftmargin}\vspace{\postGroupSkip}%
}
\renewcommand{\nomgroup}[1]{%
        \ifthenelse{\equal{#1}{P}}{%
                \groupHeading{Physical notation}%
        }{}%
        \ifthenelse{\equal{#1}{C}}{%
                \groupHeading{Coding notation}%
        }{}%
        \ifthenelse{\equal{#1}{M}}{%
                \groupHeading{Mathematical notation}%
        }{}%
}
\newcommand{\nomunit}[1]{%
        \renewcommand{\nomentryend}{\hspace*{\fill}[#1]\nolinebreak\hspace*{2cm}\mbox{}}%
}

\makenomenclature


%%%%%%%%%%%%%%%%%
% --- ARRAYS ---%
%%%%%%%%%%%%%%%%%
\usepackage{array}
\usepackage{booktabs}
\usepackage{ctable}

%%%%%%%%%%%%%%%%%%%%%%%%%%%%%%
% --- INCLUDING GRAPHICS --- %
%%%%%%%%%%%%%%%%%%%%%%%%%%%%%%
\usepackage{graphicx}
% [Information:] Set graphics path ./img/
\graphicspath{{./img/}}

% [Information:] To adjust the captions in style.tex
\usepackage{caption}
\usepackage{subcaption}


%%%%%%%%%%%%%%%%%%%%%%%%%%%%%%%%%%%
% --- PHYSICAL REPRESENTATION --- %
%%%%%%%%%%%%%%%%%%%%%%%%%%%%%%%%%%%
\usepackage{units}



%%%%%%%%%%%%%%%%%%%%%%%
% --- LOREM IPSUM --- %
%%%%%%%%%%%%%%%%%%%%%%%
\usepackage{lipsum}


\addtokomafont{disposition}{\rmfamily}


%%%%%%%%%%%%%%%%%%%%%%
% --- REFERENCES --- %
%%%%%%%%%%%%%%%%%%%%%%

\usepackage{nameref}
\newcommand*{\fullref}[1]{\hyperref[{#1}]{\autoref*{#1}}} % FOR REFERENCE BY NAME (e.g. Figure 3)

%%%%%%%%%%%%%%%%%%%%%%
% ---BIBLIOGRAPHY--- %
%%%%%%%%%%%%%%%%%%%%%%
\usepackage[backend=biber,
        style=alphabetic,
        sorting=ynt
]{biblatex}
\usepackage{csquotes}

\usepackage{hyperref}
