%%%%%%%%%%%%%%%%%%%%%%%%%%%%%%%%%%%%%%%%%%%%%%%%%%%%%%%%%%%%%%%%%%%%%%%%%%
% ------------------------------ APPENDIX ------------------------------ %
%%%%%%%%%%%%%%%%%%%%%%%%%%%%%%%%%%%%%%%%%%%%%%%%%%%%%%%%%%%%%%%%%%%%%%%%%%


\chapter{Notation \& Conventions}

\section{Natural units}
All considerations are done in natural units. This means
\begin{equation}
  \label{eq:natural-units}
  \hbar = c = 1.
\end{equation}

\section{Abstract index notation}

\subsection{Lorentz index}
The quantum field theory (QFT) is formulated on a Lorentzian manifold, namely 4-dimensional Minkowski spacetime $(\mathbb{R}^{1,3}, \eta)$, where the metric tensor $\eta$ is given by
\begin{equation}
  \label{eq:minkowski-metric}
  \left( \eta \right)_{\mu \nu} =
  \begin{pmatrix}
    1 & 0 & 0 & 0 \\
    0 & -1 & 0 & 0 \\
    0 & 0 & -1 & 0 \\
    0 & 0 & 0& -1
  \end{pmatrix}, 
\end{equation}
with $\mu,\nu = 0,1,2,3$. Further a 4-vector is just given by $x$ and the 3-vector will be clearly indicated by $\vec{p}$ for example. In abstract index notation this means $x \leftrightarrow x^\mu$, which is called the contravariant vector. The covariant vector $x_\mu$ is defined by
\begin{equation}
  x_\mu = \eta_{\mu \nu} x^\nu,  
\end{equation}
using the Einstein sum convention
\begin{equation}
  \sum_{\mu = 0}^3 \eta_{\mu \nu} x^\nu \definedas \eta_{\mu \nu} x^\nu, 
\end{equation}
where repeated indices are always summed over the whole domain of the index.