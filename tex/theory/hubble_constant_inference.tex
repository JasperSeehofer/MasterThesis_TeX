% Chapter on Inference
\chapter{Inference of the hubble constant}

\section{Dark Sirens}

Let's now start putting pieces together and construct the setup we will use to infer the hubble constant $\hubble$ from EMRI detections with LISA that do not have an electromagnetic counterpart. This means while the EMRI detection yields a measurement of the luminosity distance $\dl$, we do not have an independent redshift measurement for which one usually uses the electromagnetic counterpart of the measurement. Such measurements lacking of the electromagnetic counterpart are therefore referred to as \em{Dark Sirens}. This is where the idea of using a galaxy catalog as the counterpart comes into play.

\section{EMRI detections}

When talking about EMRI detections, form now on we mean by that the following,
\begin{definition}
    For our purposes an \emph{EMRI detection} $D$ is given by
    \begin{equation}
        \label{eq:emri-detection}
        \bm{D} \coloneqq \left \{ \dl, \vartheta, \varphi, M, \bm{\Sigma} (\dl, \vartheta, \varphi, M)\right \},
    \end{equation}
    where $\dl, \vartheta, \varphi, M, \bm{\Sigma}$ are the luminosity distance, azimuthal angle, longitudinal angle and the BH mass, covariance matrix, respectively.
\end{definition}

\section{Galaxy Catalog}
The galaxy catalog we use in the present work is the GLADE catalog . The catalog uses data from six independent astronomical catalogs. For the inference of the hubble constant we will use the galaxy catalog to cross-match EMRI detections to galaxies in the catalog as possible hosts. The information we extract from the GLADE catalog is the following:

\begin{itemize}
    \item Redshift $z$
    \item Redshift error $\sigma_z$
    \item Redshift measurement flag $z_{\text{flag}}$
    \item Galaxy stellar mass $\Mstellar$
    \item Galaxy stellar mass error $\sigma_{\Mstellar}$
    \item Azimuthal angle $\vartheta$
    \item Longitudinal angle $\varphi$
\end{itemize}

In the end we want to compare the inference of the hubble constant with and without using the stellar mass $\Mstellar$ information. So that each galaxy is available for both evaluations we remove all galaxies that do not have any information on the stellar mass $\Mstellar$. Further, we filter for galaxies that have a direct redshift measurement $z_{\text{flag}} \in {1, 3}$, where $z_{\text{flag}} = 1$ indicates a photometric redshift measurement and $z_{\text{flag}} = 3$ indicates a spectroscopic redshift measurement. By this restriction that the provided redshift is not calculated from the luminosity distance $\dl$ using a cosmological model and therefore maybe another \emph{true} value for the reduced hubble constant $\rhubble$.

\subsection{Stellar mass - central black hole mass relation}
To use the information on the stellar mass $\Mstellar$ of the galaxies in the galaxy catalog we need to relate the stellar mass to the mass of the central black hole $\Mbh$. The relation between the stellar mass of a galaxy and the mass of the central black hole has been empirically studied for example in \cite{Reines_2015}. \textbf{TODO: MORE DETAIL ON THE WORK}. The empirical relation they find is given by
\begin{equation}
    \label{eq:stellar-mass-bh-mass-relation}
    \log_{10} \left( \frac{\Mbh}{\Msol} \right) = \alpha + \beta \log_{10} \left( \frac{\Mstellar}{10^{11} \Msol} \right),
\end{equation}


\begin{definition}
    The galaxy catalog $\galcat$ is a collection of galaxies $\mathcal{G} = \left \{ \galaxy_i \right \}_{i\in\naturalnumbers}$ where each galaxy $\galaxy_i$ is given by
    \begin{equation}
        \label{eq:galaxy}
        \galaxy_i \coloneqq \left \{ z_i, \sigma_{z_i}, M_{i}, \sigma_{M_{i}}, \vartheta_i, \varphi_i \right \},
    \end{equation}
    where $z_i, \sigma_{z_i}, M_{i}, \sigma_{M_{i}}, \vartheta_i, \varphi_i$ are the redshift, redshift error, galaxy mass, galaxy mass error, azimuthal angle and longitudinal angle, respectively.
\end{definition}

\section{Galaxy Catalog cross-matching}
The concept of cross-matching the EMRI detections to galaxies in galaxy catalog is well introduced in [REFS]. The idea is that given a detection $\detection$ we can look up the \emph{completeness} of the galaxy catalog in the parameter region of $\detection$. Completeness means the fraction of galaxies that are in the catalog compared to the actual expected number or density of galaxies in given parameter region. We can then assert a likelihood to each galaxy in the catalog for it to be the \emph{host galaxy} of the EMRI detection $\detection$. In other words, since we do not exactly know the source of the EMRI event we simply consider any galaxy to be the possible host galaxy and evaluate how likely it is that the EMRI event was \emph{hosted} by this galaxy.


\section{Bayesian Inference}
The inference of the reduced hubble constant $\rhubble$ from the EMRI detections is a Bayesian inference problem. A detailed derivation of the inference is provided in \cite{Laghi_2021}. We want to infer the posterior distribution of the reduced hubble constant $\rhubble$ given the EMRI detections $\detections$, the galaxy catalog $\galcat$, a cosmological model $\cosmologicalmodel$ and background information on the inference of the hubble constant $\backgroundinformation$. The posterior probability distribution is given by Bayes' theorem
\begin{equation}
    \label{eq:bayes-theorem}
    p(\rhubble |\detections , \galcat , \cosmologicalmodel) = p(\rhubble | \galcat , \cosmologicalmodel, \backgroundinformation) \frac{p(\detections |\rhubble , \galcat , \cosmologicalmodel, \backgroundinformation) }{p(\detections | \galcat , \cosmologicalmodel , \backgroundinformation)},
\end{equation}
where the terms in the right hand side of \eqref{eq:bayes-theorem} in more detail are as follows:
\begin{description}
    \item[$p(\rhubble | \galcat , \cosmologicalmodel, \backgroundinformation)$] is the \emph{prior} distribution of the reduced hubble constant given the galaxy catalog, cosmological model and background information.
    \item[$p(\detections |\rhubble , \galcat , \cosmologicalmodel, \backgroundinformation)$] is the \emph{likelihood} of the EMRI detections given the reduced hubble constant, galaxy catalog, cosmological model and background information.
    \item[$p(\detections | \galcat , \cosmologicalmodel , \backgroundinformation)$] is the \emph{evidence} of the EMRI detections, galaxy catalog, cosmological model and background information.
\end{description}
\begin{remark}
    The evidence $p(\detections , \galcat , \cosmologicalmodel , \backgroundinformation)$ is only relevant if we want to compare different models such as cosmological models. In our case we will restrict ourselves to the flat ($k=0$) $\lamcdm$ model and therefore the evidence is not needed. Still we can use the evidence to normalize the posterior distribution.
\end{remark}

\subsection{Assumptions on EMRI detections}
The following will be assumed to hold for all (EMRI) detections:
\begin{itemize}
    \item Detections are statistically uncorrelated
    \item A detection is fully described by a point in parameter space
    \item The source of a detection is always a galaxy
    \item The BH is the central BH of the given galaxy
\end{itemize}
