\chapter{Cosmology}

\section{The Friedmann-Lemaître-Robertson-Walker metric}

The Friedmann-Lemaître-Robertson-Walker (FLRW) metric is the most general form of a homogeneous and isotropic metric. It is given by
\begin{equation}
    ds^2 = -dt^2 + a(t)^2 \left( \frac{dr^2}{1 - kr^2} + r^2 d\Omega^2 \right),
\end{equation}
where $a(t)$ is the scale factor, $k$ is the curvature of the universe, and $d\Omega^2 = d\theta^2 + \sin^2\theta d\phi^2$ is the solid angle element. The curvature parameter $k$ can take the values $k = 0, \pm 1$, corresponding to a flat, closed, or open universe, respectively. The scale factor $a(t)$ is a function of time and describes the expansion of the universe. The Hubble parameter $H(t)$ is defined as
\begin{equation}
    H(t) = \frac{\dot{a}(t)}{a(t)},
\end{equation}
where the dot denotes a derivative with respect to time. The Hubble parameter is a measure of the rate of expansion of the universe.

\section{The Friedmann equations}

The Friedmann equations describe the evolution of the scale factor $a(t)$ in an FLRW universe. They are given by
\begin{align}
    \left( \frac{\dot{a}}{a} \right)^2 & = \frac{8\pi G}{3} \rho - \frac{k}{a^2}, \label{eq:friedmann1} \\
    \frac{\ddot{a}}{a}                 & = -\frac{4\pi G}{3} (\rho + 3p), \label{eq:friedmann2}
\end{align}
where $\rho$ is the energy density and $p$ is the pressure. The first Friedmann equation \eqref{eq:friedmann1} is a statement of energy conservation, while the second Friedmann equation \eqref{eq:friedmann2} is a statement of momentum conservation. The Friedmann equations can be derived from the Einstein field equations by assuming a homogeneous and isotropic universe.

\section{The hubble constant and its tension}


\section{measuring the hubble constant}

\section{\texorpdfstring{$\Lambda$CDM model}{ΛCDM model}}

