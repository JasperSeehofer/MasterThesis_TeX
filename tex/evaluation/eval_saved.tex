{sec:simulated-detections}
We have simulated a total number $N_{\detections} = 2254$ that were detected with an optimal SNR above the threshold $\text{SNR}_{\text{th}} = 20$. The average Cramér-Rao bounds are shown in \fullref{fig:galaxy-catalog-only-cramer-rao-bounds}. We can see that the precision on the intrinsic parameters ($\Mz, \mu, a, p_0, e_0$) is a lot higher. We then filtered the detections to only include those with a relative luminosity distance uncertainty $\sigma_{\dl} / \dl < 0.05$ [REF to consideration that precise measurements determine the shape and the goodness of the posterior.]. This leaves us with $N_{\detections} = 312$ detections. The distribution of the relative luminosity distance uncertainty is shown in \fullref{fig:rel-luminosity-distance-uncertainty}. For this subset of detections $\detections$ we run the inference procedure to compute the posterior probability distribution of $\rhubble$ as derived in \fullref{ch:inference-of-hubble}.

For all posterior distributions we have used the freedom of choice in the normalization constant from \fullref{eq:bayes-theorem-no-evidence} to normalize the distribution to a maximum of $1$.
In \fullref{fig:posterior-rhubble-single-detections} we show the posterior probability distribution of $\rhubble$ for the single detections without and with the usage of the black hole mass $\Mz$ measurement and the galaxy catalog as ground truth. We can see that the usage of the black hole mass $\Mz$ measurement significantly improves the precision of the posterior probability distribution of $\rhubble$ in terms of the precision but also in consistency, as nearly every detection yields a good estimation of $\rhubble$. In \fullref{fig:posterior-rhubble} we show the posterior probability distribution of $\rhubble$ using all detections. Fitting a gaussian distribution to the posterior probability distribution of $\rhubble$ we find the following results:
\begin{equation}
    \label{eq:galaxy-catalog-only-results}
    \boxed{
        \begin{aligned}
            \hat{\rhubble}           & = 0.736 \pm 0.011, \\
            \hat{\rhubble}_\text{bh} & = 0.730 \pm 0.001.
        \end{aligned}
    }
\end{equation}
The posterior probability distribution of $\rhubble$ for random subsets with 100 and 25 detections of the simulated detections with the measured black hole mass $\Mz$ is shown in \fullref{fig:posterior-rhubble-subsets} and \fullref{fig:posterior-rhubble-subsets-25}, respectively. We can see that the posterior probability distribution of $\rhubble$ is consistent across different subsets of detections. This is a good indicator that the inference procedure is robust and that the posterior probability distribution of $\rhubble$ is not biased by the choice of detections.
\subsubsection{25 detections}
Even more remarkable are the results of the subset with 25 detections using $\Mbh$ in the inference. While the inference of $\rhubble$ starts to break down in the inference without $\Mbh$ for subsets with 25 detections, the usage of $\Mbh$ in the inference procedure yields consistent results across different subsets. Hence, even if very few detections will be available from LISA, the usage of the black hole mass $\Mbh$ in the inference procedure will yield consistent results for the Hubble constant $\rhubble$ with a precision of roughly
\begin{equation}
    \frac{\Delta \hat{\rhubble}_{\text{bh}}}{\hat{\rhubble}_{\text{bh}}} \approx 0.5\unit{\%},
\end{equation}
compared to the precision of roughly
\begin{equation}
    \frac{\Delta \hat{\rhubble}}{\hat{\rhubble}} \approx 7\unit{\%},
\end{equation}
without the usage of the black hole mass $\Mbh$ in the inference procedure.
    [TODO: figure with single likelihood visualized and normalization factor]

\begin{figure}
    \centering
    \includegraphics[width=0.8\textwidth]{mean_bounds.png}
    \caption[Average Cramér-Rao bounds]{The average Cramér-Rao bounds for the parameters of the EMRI signal.}
    \label{fig:galaxy-catalog-only-cramer-rao-bounds}
\end{figure}

\begin{figure}
    \centering
    \includegraphics[width=0.8\textwidth]{error_dist.png}
    \caption[Relative luminosity distance uncertainty]{The distribution of the relative luminosity distance uncertainty $\sigma_{\dl} / \dl$ for the simulated detections.}
    \label{fig:rel-luminosity-distance-uncertainty}
\end{figure}

\begin{figure}
    \centering
    \includegraphics[width=\textwidth]{bayesian_statistics_event_posteriors.png}
    \includegraphics[width=\textwidth]{bayesian_statistics_event_posteriors_with_bh_mass.png}
    \caption[Posterior probability distribution of single detections]{The posterior probability distribution of $\rhubble$ for the single detections without (top) and with (bottom) the usage of the black hole mass $\Mz$ measurement and the galaxy catalog as ground truth. The color indicates the relative luminosity distance uncertainty $\sigma_{\dl} / \dl$. The vertical blue dashed line indicates the true value of $\rhubble = 0.73$.}
    \label{fig:posterior-rhubble-single-detections}
\end{figure}

\begin{figure}
    \centering
    \includegraphics[width=\textwidth]{bayesian_statistics.png}
    \caption[Posterior probability distribution of $\rhubble$]{The posterior probability distribution of $\rhubble$ for the simulated detections with and without using the measured black hole mass $\Mz$.The vertical green dashed line indicates the true value of $\rhubble = 0.73$.}
    \label{fig:posterior-rhubble}
\end{figure}

\begin{figure}
    \centering
    \includegraphics[width=\textwidth]{bayesian_statistics_event_posteriors_subsets.png}
    \caption[Posterior probability distribution of $\rhubble$ of subsets (100 detections)]{The posterior probability distribution of $\rhubble$ for random subsets with 100 detections of the simulated detections with the measured black hole mass $\Mz$. The vertical green dashed line indicates the true value of $\rhubble = 0.73$. The colors indicate the different subsets.}
    \label{fig:posterior-rhubble-subsets}
\end{figure}

\begin{figure}
    \centering
    \includegraphics[width=\textwidth]{bayesian_statistics_event_posteriors_subsets_25.png}
    \caption[Posterior probability distribution of $\rhubble$ of subsets (25 detections)]{The posterior probability distribution of $\rhubble$ for random subsets with 25 detections of the simulated detections with the measured black hole mass $\Mz$. The vertical green dashed line indicates the true value of $\rhubble = 0.73$. The colors indicate the different subsets.}
    \label{fig:posterior-rhubble-subsets-25}
\end{figure}
