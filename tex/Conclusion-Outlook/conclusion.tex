\chapter{Conclusion \& Outlook}
The first part of the conclusion shall be dedicated to the \emph{commonly} used notions and notations in the field of DIS and SIDIS. Despite the fact of being able the formulate a very compact representation of the different roles that appear in the play of SIDIS as can be seen in the SIDIS cross section in \fullref{eq:SIDIS-cs-final-factor} and the elegance of isolating the data of interest for the measurements into the structure function $\mathcal{F}_{T/L}$, there seems to be a whole collection of different structure functions, each with a different normalization or defined in a slightly different manner, making it very difficult for physicists new to the field unpleasant to grasp the ideas and being able to relate them between different papers, works and even textbooks. In my opinion and in retro-perspective I would had preferred to have used Altarelli's \cite[Eq. (36)]{ALTARELLI1979461} definition of the hadronic tensor to not end up with the mass factor for the projectors as we had to introduce in \fullref{eq:projections-t-l}.  Since the actual goal is the information extraction about \emph{any} pair of structure functions from measurements and not having an overview what factor is needed to obtain the same result as given in a reference book, it would be unburdening if there would be a commonly used definition of the structure functions and accordance in the definition of variables such as the SIDIS variables $x,y,z$, of which no unique definition exists. A defined $x$ will turn out the be the $z$ of another work and so on. After all, really diving into the theoretical toolkit of SIDIS is very enlightening and the puzzling described above can motivate a meticulous way of reading and working the notions out for one self. \\
Another experience worth noting that has been gained from this work, is the calculations needed for this subject at first help understanding the kinematics of the processes and questioning every appearing variable relation but on the long run it turns out to be very inefficient to be a human and not a computer in the sense that the sources of errors in the computations never seem to have an end even though the calculations are not difficult. This develops a understanding of why one uses Mathematica packages such as \emph{Tracer} to outsource this and have more time to think about the implications of what one has computed or rather obtained.\\
Ultimately SIDIS appears to be a very useful theoretical approach of extracting more information from experiments about the structure of the particles such as protons and the distributions of the constituent partons. Especially, using the cross section predictions as provided in this paper one can reverse engineer contributions to the cross section that are purely related to the gluon distribution within the proton. In more detail even specific gluons, those with heavy quarks contained in them. Thus having both the inclusive DIS which tries to predict the scattering at a global level and the theoretical preliminary work that is being done numerously for specific contributions can help to unravel the whole appearance and dynamics of QCD and its manifestation in the observable regime, the cross sections. \\
Even though there seems to be a factor of $\frac{1}{8 \pi}$ that either is wrong in the results in the present thesis or is due to different normalizations when referring to other papers or works it was possible to develope the formalism of SIDIS in a more or less compact way and helped me understanding the subject alot. Hopefully it will be instructive to other people as well.

\section{Outlook}
Having obtained \fullref{eq:SIDIS-cs-final-factor} one could start doing several additional considerations, starting with the integrations inherited from the structure functions and really obtaining the dependencies on the mass terms. Further we could also go on with more detailed calculations such as a polarized calculation of the SIDIS cross section, i.e.\ either the electron beam or the proton beam could be polarized, which would allow us to not do all the averaging we had to do due to the lack of knowledge about the spin and polarization. It would also be possible to now grab some measurements of the given process and test the given structure functions and see whether they math the experiment or if there are deviations from it. There is definitely a lot to come with new hadron colliders in planning and more and more detailed evaluations for a deeper understanding of the subatomic structure and dynamics of the proton and other particles.

\section{Acknowledgement}
I am very thankful for every answered question and support whenever it was needed by Prof. Dr. Werner Vogelsang. It was definetly a great choice to do the Bachelor thesis with you as a supervisor. I am looking forward to any other possiblity of working in your group!