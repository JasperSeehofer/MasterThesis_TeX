\chapter{Conclusion and Outlook}
Most of the discussion of the results has already been done in the previous chapter. To summarize, while we were able to reproduce and validate the parameter estimations of the parameters measured by LISA for an EMRI event using the template waveform provided by \emph{FEW} \cite{Katz_2021,Chua_2021} as predicted by \cite{PhysRevD.95.103012} and obtain a result for the constraint of the reduced Hubble constant $\rhubble$ using the normalization scenario of the posterior distribution as derived in \cite{Chen_2022} with a $1-\sigma$ accuracy of $1.8\%, 4\%$ for the inference models using the measured redshifted MBH mass $\Mz$ and without using it, respectively, with relative mean deviations w.r.t. to the injected $\rhubbletrue=0.73$ of $1.6\%, -2.4\%$, which are good results in the sense that they are close but still approximately doubled to what \cite{10.1093/mnras/stab2741} predicted for a 90\% confidence level in the case of the M1 model. It is obvious, that we can not expect the same results, because the posterior distributions do not follow the astrophysical prediction of EMRI events and are biased as has been discussed in \fullref{ch:galaxy-catalog-as-ground-truth}. Nevertheless, one can already see an improvement in the results when applying the more detailed construction of the posterior distribution using the information of the measured redshifted MBH mass $\Mz$. Considering this additional dimension in the parameter space it is possible to suppress more possible hosts in their likelihood and in the best case can remove galaxies that are on the line of sight, because this is what the usual gaussian peak in the likelihood w.r.t. $(\dl - \dl(z, \rhubbletrue))^2$ is sensible to and can therefore prevent high likelihoods for the wrong galaxies. In short, the results are promising to push the accuracy of constraining the Hubble constant with EMRI events to a new level.

\section{Outlook}
The next steps building on the present work are to first resolve the issue of the bias occurring in both scenarios of the normalization to have comparable results to recent predicted constraints on the Hubble constant. Further, the inference can be tested on more realistic scenarios using predicted EMRI distributions to sample the events and adopting to an evaluation that takes the galaxy catalog incompleteness into account. This should be straight forward, as the tools have already been developed (for example see \cite{Fishbach_2019}). This has already been implemented but the evaluation was not finished at the time of writing this thesis. In general, the inference of the Hubble constant with EMRI dark sirens will always heavily depend on the galaxy catalog and the accuracy of the redshift measurements and mass measurements. So any improvement in these areas will directly improve the inference of the Hubble constant.

\section{Acknowledgement}
I want to thank my supervisor, Prof. PhD Lavinia Heisenberg and PhD Do\u{g}a Veske for their guidance and support during my thesis and many illuminating discussions. I would also like to thank my family and friends for their support and encouragement during my studies. The authors acknowledge support by the state of Baden-Württemberg through bwHPC.
